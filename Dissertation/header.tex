\documentclass[twoside,openright,titlepage,fleqn,
	headinclude,11pt,a4paper,BCOR5mm,footinclude
	]{scrbook}
%\documentclass[a4paper,11pt]{book}
\usepackage[utf8]{inputenc}
%\usepackage[latin1]{inputenc}
\usepackage{lmodern}
\usepackage{microtype}
\usepackage[T1]{fontenc}
%\usepackage[italian]{babel}
%\usepackage{graphicx}
\usepackage{subfiles}
\usepackage{listings}
%\usepackage{pgf}
\usepackage{xparse}
\usepackage{float}
\usepackage{color}
%\usepackage{xcolor}

%pacchetti del modello per la tesi
%--------------------------------------------------------------
\usepackage[square,numbers]{natbib} 
\usepackage[fleqn]{amsmath}  
\usepackage{dia-classicthesis-ldpkg} 
% Options for classicthesis.sty:
% tocaligned eulerchapternumbers drafting linedheaders 
% listsseparated subfig nochapters beramono eulermath parts 
% minionpro pdfspacing
\usepackage[eulerchapternumbers,subfig,beramono,eulermath,
	parts]{classicthesis}
%--------------------------------------------------------------
\usepackage{tikz}
\usetikzlibrary{shapes, chains, scopes, shadows, positioning, arrows,
  decorations.pathmorphing, calc, mindmap, petri}

\usepackage{cleveref}

%comandi per modello
%--------------------------------------------------------------
\newcommand{\myTitle}{B-Spline methods for designing smooth spatial
  path with obstacle avoidance\xspace}
% use the right myDegree option
\newcommand{\myDegree}{Master's degree in computer science\xspace}
\newcommand{\myName}{Stefano Martina\xspace}
\newcommand{\myMail}{stefano.martina@stud.unifi.it}
\newcommand{\myProf}{Alessandra Sestini\xspace}
%\newcommand{\myOtherProf}{Carlotta Giannelli\xspace}
\newcommand{\mySupervisor}{Carlotta Giannelli\xspace}
\newcommand{\myFaculty}{
	School of mathematical, physical and natural sciences\xspace}
\newcommand{\myDepartment}{
	Department of mathematics and computer science\xspace}
\newcommand{\myUni}{\protect{
	University of Florence}\xspace}
\newcommand{\myLocation}{Florence\xspace}
\newcommand{\myAcademicYear}{2015/2016\xspace}
\newcommand{\myYear}{2016\xspace}
\newcommand{\myVersion}{Version 0.1\xspace}
\newcommand{\mycopyright}{
  this work is licensed under a
  \href{http://creativecommons.org/licenses/by-sa/4.0/}{Creative
    Commons Attribution-ShareAlike 4.0 International License \includegraphics[width=1cm]{logo/logoCC.png}}\xspace}
\newlength{\abcd} % for ab..z string length calculation
% how all the floats will be aligned
\newcommand{\myfloatalign}{\centering} 
\setlength{\extrarowheight}{3pt} % increase table row height
\captionsetup{format=hang,font=small}
%--------------------------------------------------------------
% Layout setting
%--------------------------------------------------------------
\usepackage{geometry}
\geometry{
	a4paper,
	ignoremp,
	bindingoffset = 1cm, 
	textwidth     = 13.5cm,
	textheight    = 21.5cm,
	lmargin       = 3.5cm, % left margin
	tmargin       = 4cm    % top margin 
}
%--------------------------------------------------------------

%comando per impostazioni float di default
\makeatletter
\def\fps@figure{!htbp}
\def\fps@table{!htbp}
\def\fps@code{!htbp}
\makeatother

%cambia comportamento delle description
\renewcommand{\descriptionlabel}[1]{\hspace{2em}\hspace{\labelsep}\textbf{#1}}


%definisce un nuovo ambiente float per il codice
\newfloat{code}{!htbp}{}
\floatname{code}{Codice}

%comando per dimensioni testo
\newcommand{\dimg}{\tiny}
\newcommand{\codg}[1]{\dimg \unicocodet{#1}}
%  \lstinline[basicstyle=\dimg\ttfamily\bfseries,breaklines=true]|#1|


\newcommand{\dims}{\scriptsize}
\newcommand{\cods}[1]{\dims\unicocodet{#1}}
%  \lstinline[basicstyle=\dims\ttfamily\bfseries,breaklines=true]|#1|

\NewDocumentEnvironment{myfig}{mm}{
  \begin{figure}
    \begin{center}
    }{
    \end{center}
    \caption{#1}
    \label{#2}
  \end{figure}
}

%comando per inserire una immagine
\NewDocumentCommand{\image}{ommm}{
  \begin{figure}
    \begin{center}
      \begin{tikzpicture}
        \node [rectangle, rounded corners=2pt, inner sep = 0.3cm, drop shadow, draw=black!50, fill=white] {
          \IfNoValueTF{#1}
                      {\includegraphics[width=.9\textwidth]{#2}}
                      {\includegraphics[width=#1]{#2}}
        };
      \end{tikzpicture}
    \end{center}
    \caption{#3}
    \label{#4}
  \end{figure}
}

\NewDocumentCommand{\imageDouble}{ommmm}{
  \begin{figure}
    \begin{center}
      \begin{tikzpicture}
        \node [rectangle, rounded corners=2pt, inner sep = 0.3cm, drop shadow, draw=black!50, fill=white] {
          \IfNoValueTF{#1}
                      {\includegraphics[width=.42\textwidth]{#2}}
                      {\includegraphics[width=#1]{#2}}
        };
      \end{tikzpicture}
      \begin{tikzpicture}
        \node [rectangle, rounded corners=2pt, inner sep = 0.3cm, drop shadow, draw=black!50, fill=white] {
          \IfNoValueTF{#1}
                      {\includegraphics[width=.42\textwidth]{#3}}
                      {\includegraphics[width=#1]{#3}}
        };
      \end{tikzpicture}
    \end{center}
    \caption{#4}
    \label{#5}
  \end{figure}
}

\NewDocumentCommand{\imager}{ommm}{
  \begin{figure}
    \begin{center}
      \begin{tikzpicture}
        \node [rectangle, rounded corners=2pt, inner sep = 0.3cm, drop shadow, draw=black!50, fill=white, rotate = 90] {
          \IfNoValueTF{#1}
                      {\includegraphics[height=13cm]{#2}}
                      {\includegraphics[height=#1]{#2}}
        };
      \end{tikzpicture}
    \end{center}
    \caption{#3}
    \label{#4}
  \end{figure}
}

%comando per inserire un link
\newcommand{\link}[1]{\unicocode{#1}}


%comando per inserire un file
\newcommand{\file}[1]{\unicocode{#1}}

\newcommand{\bigO}{\ensuremath{\mathcal{O}}}

\def\transW{8mm}
\def\transH{2mm}

\tikzstyle{obstacle}=[fill=green, draw=black]
\tikzstyle{convexHull}=[fill=blue!30]
\tikzstyle{convexHullBord}=[color=black, dash pattern=on 3pt off 3pt]
\tikzstyle{controlPoly}=[color=black, line width=0.25mm]
\tikzstyle{controlPolyTract}=[color=black, line width=0.25mm, dash pattern=on 3pt off 3pt]
\tikzstyle{controlPolyTractHigh}=[color=red, line width=0.25mm, dash pattern=on 3pt off 3pt]
\tikzstyle{obstacleTract}=[draw=black, line width=0.25mm, dash pattern=on 3pt off 3pt]
\tikzstyle{spline}=[color=red, line width=0.5mm]
\tikzstyle{controlVert}=[color=green, draw=black]
\tikzstyle{controlVertHigh}=[color=red, draw=black]
\tikzstyle{obstaclePoint}=[color=red, draw=black]
\tikzstyle{site}=[color=blue, draw=black]
\tikzstyle{siteHigh}=[color=red, draw=black]
\tikzstyle{textArrow}=[draw=red, line width=0.5mm, ->]

\colorlet{mmcb}{black!70}
\colorlet{mmc1}{red!80}
\colorlet{mmc2}{blue!80}
\colorlet{c1}{green!20}
\colorlet{c2}{blue!10}
\colorlet{c3}{yellow!10}
\colorlet{c4}{red!10}
\colorlet{drawColor}{black!80}
\colorlet{commentColor}{green!70!black!90}
\colorlet{codeBgColor}{yellow!50}
\colorlet{bashBgColor}{green!50}

\tikzset{onslide/.code args={<#1>#2}{%
  \only<#1>{\pgfkeysalso{#2}} % \pgfkeysalso doesn't change the path
}}
\tikzset{temporal/.code args={<#1>#2#3#4}{%
  \temporal<#1>{\pgfkeysalso{#2}}{\pgfkeysalso{#3}}{\pgfkeysalso{#4}} % \pgfkeysalso doesn't change the path
}}

\tikzstyle{alertStar}=[circle, decorate, decoration={zigzag,segment length=3.12mm,amplitude=1mm}, align=center, drop shadow, draw=drawColor, fill=white]
\tikzstyle{oval}=[ellipse, align=center, drop shadow, draw=drawColor, fill=white]
\tikzstyle{rect}=[rectangle, rounded corners=2pt, align=center, drop
shadow, draw=drawColor, fill=white]
\tikzstyle{arrow}=[->, very thick, >=stealth', draw=black!80]
\tikzstyle{myMindmap}=[mindmap,
every node/.style={concept, minimum size=5mm, text width=5mm}, 
% every child/.style={level distance=10mm, concept color=mmcb}
level 1/.append style={level distance=10mm,sibling angle=45},
level 2/.append style={level distance=10mm,sibling angle=45},
level 3/.append style={level distance=10mm,sibling angle=45}
]
\tikzstyle{myPlace} = [place, very thick, draw=drawColor, fill=white, drop shadow]
\tikzstyle{transExpH} = [transition, very thick, draw=drawColor, fill=white, drop
shadow, minimum width=\transW, minimum height=\transH]
\tikzstyle{transExpV} = [transition, very thick, draw=drawColor, fill=white, drop
shadow, minimum width=\transH, minimum height=\transW]
\tikzstyle{transDetH} = [transition, very thick, draw=drawColor, fill=black, drop shadow, minimum width=\transW, minimum height=\transH]
\tikzstyle{transDetV} = [transition, very thick, draw=drawColor, fill=black, drop shadow, minimum width=\transH, minimum height=\transW]
\tikzstyle{pre}=[<-, very thick, >=stealth', draw=drawColor]
\tikzstyle{preN}=[<-, very thick, >=o, draw=drawColor]
\tikzstyle{post}=[->, very thick, >=stealth', draw=drawColor]
\tikzstyle{highlight}=[draw=red]
\lstdefinestyle{customPython}{
   language=Python,
   % basicstyle=\small\ttfamily\bfseries,
   basicstyle=\tiny\ttfamily,
   keywordstyle=\color{blue}\ttfamily,
   stringstyle=\color{red}\ttfamily,
   commentstyle=\color{green}\ttfamily,
   morecomment=[l][\color{magenta}]{\#},
   % breaklines=false,
   breaklines=true, breakatwhitespace=true,
   postbreak=\raisebox{0ex}[0ex][0ex]{\ensuremath{\color{red}\hookrightarrow\space}},
   frameround=fttt,
   frame=trBL,
   backgroundcolor=\color{yellow!20},
   numbers=left,
   stepnumber=1,    
   firstnumber=1,
   numberfirstline=true,
   numberstyle=\tiny\color{black!50},
   xleftmargin=1.75em,
   framexleftmargin=2.1em,
   % rulesepcolor=\color{gray},
   rulecolor=\color{black}
   % linewidth=8cm,
}

\lstdefinestyle{customInlinePython}{
   language=Python,
   % basicstyle=\small\ttfamily\bfseries,
   basicstyle=\ttfamily,
   keywordstyle=\color{blue}\ttfamily,
   stringstyle=\color{red}\ttfamily,
   commentstyle=\color{green}\ttfamily,
   morecomment=[l][\color{magenta}]{\#}
}

\lstnewenvironment{pblock}[1][]
{
  \lstset{
    style=customPython,
    #1
  }
}{}

\newcommand{\pfile}[2][]{
  \lstinputlisting[style=customPython, title={\texttt{\detokenize{#2}}}, #1]{#2}
}

\newcommand{\pp}[2][]{\lstinline[style=customInlinePython,#1]`#2`}
  %\colorbox{codeBgColor}{
  %  \lstinline[style=customPython,#1]`#2`
  %}
%}

\graphicspath{{img/}}
\lstset{inputpath=src/}

%% \definecolor{links}{HTML}{2A1B81}
%% \hypersetup{colorlinks,linkcolor=links,urlcolor=links}

%% \definecolor{links}{HTML}{2A1B81}
%% \hypersetup{colorlinks,linkcolor=,urlcolor=links}

\newcommand{\me}{\ensuremath{\mathrm{e}}}
\newcommand{\md}{\ensuremath{\mathrm{d}}}
\newcommand{\tc}{\ensuremath{\mathrm{t.c.:}\quad}}
\newcommand{\expected}[1]{\ensuremath{\mathrm{\textbf{E}}\left[#1\right]}}
\newcommand{\variance}[1]{\ensuremath{\mathrm{\textbf{Var}}\left(#1\right)}}
\newcommand{\prob}[1]{\ensuremath{\mathrm{\textbf{P}}\left(#1\right)}}
%\newcommand{\max}[1]{\ensuremath{\mathrm{max}\left(#1\right)}}
\newcommand{\abs}[1]{\ensuremath{\left|#1\right|}}
\newcommand{\mR}{\ensuremath{\mathbb{R}}}
\newcommand{\mN}{\ensuremath{\mathbb{N}}}


%%% Local Variables:
%%% mode: latex
%%% TeX-master: "dissertation"
%%% End:
